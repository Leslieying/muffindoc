% cGENIE QuickStartGuide document

% Andy Ridgwell, August 2014
%
% ---------------------------------------------------------------------------------------------------------------------------------
% ---------------------------------------------------------------------------------------------------------------------------------

\documentclass[10pt,twoside]{article}
\usepackage[paper=a4paper,portrait=true,margin=1.5cm,ignorehead,footnotesep=1cm]{geometry}
\usepackage{graphicx}
\usepackage{hyperref}
\usepackage{paralist}
\usepackage{caption}
\usepackage{float}
\usepackage{wasysym}
\usepackage{amssymb}


\linespread{1.1}
\setlength{\pltopsep}{2.5pt}
\setlength{\plparsep}{2.5pt}
\setlength{\partopsep}{2.5pt}
\setlength{\parskip}{2.5pt}

%\addtolength{\oddsidemargin}{1.0cm}
%\addtolength{\bottommargin}{1.0cm}

\title{cGENIE Quick-start Guide: `muffin' version [Mac]}
\author{Ben Ward}
\date{\today}
\usepackage[normalem]{ulem}

\begin{document}

%=================================================================================================================================
%=== BEGIN DOCUMENT ==============================================================================================================
%=================================================================================================================================

\maketitle

%---------------------------------------------------------------------------------------------------------------------------------
%--- Quick-start guide for cGENIE ---------------------------------------------------------------------------------
%---------------------------------------------------------------------------------------------------------------------------------

\noindent \textbf{This is the Quick-start Guide for installing cGENIE-muffin on a Mac.}

\noindent To install the muffin release of cGENIE on a Mac you will need a number of software packages, including Fortran, C++ and NetCDF. The best way to get hold of these is via a package management system, such as Homebrew (\href{}{https://brew.sh}) or MacPorts (\href{}{https://www.macports.org}). This guide is based on Homebrew, because it is slightly more user friendly than MacPorts, and is kept more up-to-date with changes in the Apple operating system. (If you are already a MacPorts user, please see Appendix~A.)

\begin{compactenum}

\item First of all, you will need XCode, which can be downloaded from the app store, or here... 
\\\href{https://developer.apple.com/xcode/downloads}{https://developer.apple.com/xcode/downloads} \\After installing XCode, it is necessary to enable command line tools, by entering at the command line...\\
\texttt{xcode-select --install} 

% I think XCode automatically comes with Homebrew and command line tools enabled
\item Get Homebrew by pasting the following at the terminal command line...
\\ \texttt{/usr/bin/ruby -e "\$(curl -fsSL https://raw.githubusercontent.com/Homebrew/install/master/install)"}
\\ Next, type `\texttt{brew doctor}'. This should tell you ``\texttt{your system is ready to brew}''. If it doesn't, see Appendix~B. 

\item Install Fortran, C++, NetCDF and some other useful libraries at the command line (using Homebrew) as follows:

% brew tap homebrew/science % deprecated in homebrew
\vspace{-5pt}\begin{verbatim}
brew install cmake
brew install gcc
brew install hdf5
brew install netcdf
brew install wget
\end{verbatim}

\item Get hold of a (read-only) copy of the current 'muffin' branch of \textit{c}GENIE source code via the command:
\vspace{-5pt}\begin{verbatim}
svn co https://svn.ggy.bris.ac.uk/subversion/genie/branches/cgenie.muffin
--username=genie-user cgenie.muffin
\end{verbatim}\vspace{-5pt}
for the 'head' (current development version).
NOTE: All this must be typed continuously on ONE LINE, with a S P A C E before `\texttt{--username}', and before `\texttt{cgenie}'.
If you are asked about a `certificate', enter `p' to accept `permanently' (`a' makes it grumpy). Unless you have logged onto the \texttt{svn} server before from your computing account, you be asked for a password -- it is \texttt{g3n1e-user}.


\item Check your netcdf version number by entering \texttt{brew info netcdf}\\
This will return several lines, but the key one gives the netcdf path, and should look something like:
\vspace{-5pt}\begin{verbatim}
/usr/local/Cellar/netcdf/4.6.1_2 (84 files, 6.2MB) *
\end{verbatim}\vspace{-5pt}
(This was from my most recent install, with version \texttt{4.6.1\_2})

\item Adjust the cGENIE environment variables for your machine and netcdf installation by editing
\\\texttt{cgenie.muffin/genie-main/user.mak}, setting:
\vspace{-5pt}\begin{verbatim}
MACHINE=OSX
\end{verbatim}\vspace{-5pt}
and
\vspace{-5pt}\begin{verbatim}
NETCDF_DIR=/usr/local/Cellar/netcdf/4.6.1_2
\end{verbatim} \vspace{-5pt}
to reflect your netcdf path (Step 5).

\item Finally, in cgenie.muffin/genie-main/makefile.arc, comment out:
\vspace{-5pt}\begin{verbatim}
#NETCDF=
   $(LIB_SEARCH_FLAG)$(PATH_QUOTE)$(NETCDF_DIR)/lib$(PATH_QUOTE) $(LIB_FLAG)$(NETCDF_NAME)
\end{verbatim}\vspace{-5pt}
and un-comment 
\vspace{-5pt}\begin{verbatim}
NETCDF_NAMEF=$(NETCDF_NAME)f
NETCDF=$(LIB_SEARCH_FLAG)$(PATH_QUOTE)$(NETCDF_DIR)/lib$(PATH_QUOTE) 
   $(LIB_FLAG)$(NETCDF_NAME) $(LIB_FLAG)$(NETCDF_NAME) $(LIB_FLAG)$(NETCDF_NAMEF)
\end{verbatim}

\item To test the code installation -- change directory to \texttt{cgenie.muffin/genie-main} and type:
\vspace{-5pt}\begin{verbatim}
make testbiogem
\end{verbatim}\vspace{-5pt}
This compiles a carbon cycle enabled configuration of \textit{c}GENIE and runs a short test, comparing the results against those of a pre-run experiment (also downloaded alongside the model source code). It serves to check that you have the software environment correctly configured. If you are unsuccessful here ... double-check the software and directory environment settings in \texttt{user.mak} (or \texttt{user.sh}) and for a netCDF error, check the value of the \texttt{NETCDF\_DIR} environment variable. (Refer to the User Manual for addition fault-finding tips.) If environment variables are changed: before re-trying the test, you will need to type:
\vspace{-5pt}\begin{verbatim}
make cleanall
\end{verbatim}\vspace{-5pt}

\end{compactenum}

\noindent That is is for the basic installation. To run the model it is a simple matter of calling the  '\texttt{runmuffin.sh}'  shell script from \texttt{genie-main} and supplying a couple of parameter values, e.g.:
\vspace{-5pt}\small\begin{verbatim}./runmuffin.sh cgenie.eb_go_gs_ac_bg.worjh2.ANTH / EXAMPLE.worjh2.Caoetal2009.SPIN 10000\end{verbatim}\normalsize\vspace{-5pt}
Refer to the \textit{c}GENIE \texttt{User\_manual} for more information regarding installing, running, and analyzing model output, and \textit{c}GENIE \texttt{Examples} for more information on this specific example.\footnote{latex source for all the documents can be found in the \texttt{genie-docs} directory, with recent PDF versions at www.seao2.info/mycgenie.html.} \uline{Read the \textit{c}GENIE \texttt{READ-ME}}.



\newpage
\section*{Appendix A}

You are here because you have already installed MacPorts, presumably by following the instructions here:
\\\href{https://www.macports.org/install.php}{https://www.macports.org/install.php}.
You now have two options, either remove MacPorts entirely, replacing it with Homebrew, or install the required packages through MacPorts and a number of precompiled binaries.

\subsection*{Option 1: Remove MacPorts from your system... (to be replaced by Homebrew)}

\begin{compactenum}

\item Back up your system (i.e. using Time Machine).

\item To uninstall MacPorts, enter at the terminal:\\
\texttt{sudo port -f uninstall installed}
 
Then remove everything that is left from MacPorts: \\
\texttt{sudo rm -rf /opt/local} \\
\texttt{sudo rm -rf /Applications/DarwinPorts} \\
\texttt{sudo rm -rf /Applications/MacPorts} \\
\texttt{sudo rm -rf /Library/LaunchDaemons/org.macports.*} \\
\texttt{sudo rm -rf /Library/Receipts/DarwinPorts*.pkg}\\
\texttt{sudo rm -rf /Library/Receipts/MacPorts*.pkg} \\
\texttt{sudo rm -rf /Library/StartupItems/DarwinPortsStartup} \\
\texttt{sudo rm -rf /Library/Tcl/darwinports1.0} \\
\texttt{sudo rm -rf /Library/Tcl/macports1.0} \\
\texttt{sudo rm -rf \textasciitilde/.macports}

Note that the \texttt{sudo} command is inserted before the \texttt{rm} (i.e. remove) command in order to enable the correct permissions.

\item You may now continue with your installation as described in the main text. You may have to delete some files (using \texttt{sudo rm}), as recommended by \texttt{brew doctor}. 

\end{compactenum}


\subsection*{Option 2: Install required packages through MacPorts and precompiled binaries...}

\begin{compactenum}

\item First of all, synchronize your installation of MacPorts:
\\\texttt{sudo port -v selfupdate} 

\item Then install Netcdf and related C\(^{++}\) and Fortran libraries at the command line using MacPorts, as follows:

\vspace{-5pt}\begin{verbatim}
sudo port install netcdf
sudo port install netcdf-cxx
sudo port install netcdf-fortran
\end{verbatim}

\item Download precompiled fortran and C\(^{++}\)  binaries appropriate to your operating system (El Capitan \& Sierra, etc.) from \href{http://hpc.sourceforge.net}{http://hpc.sourceforge.net}. Install as follows, amending the file number to match your version of OSX (e.g. El Capitan \& Sierra are associated with the 7.1 binaries).

\vspace{-5pt}\begin{verbatim}
cd ~/Downloads/
gunzip gcc-7.1-bin.tar.gz
sudo tar -xvf gcc-7.1-bin.tar -C /.
gunzip gfortran-7.1-bin.tar.gz 
sudo tar -xvf gfortran-7.1-bin.tar -C /.
\end{verbatim}

If your operating system is not listed here, you will either have to wait until it is, or install Homebew.

\end{compactenum}


\section*{Appendix B}

Errors identified by `\texttt{brew doctor}' are most likely associated with some incompatible files in your software libraries, perhaps from a previous installation of MacPorts. Try to follow the suggestions given to you by `\texttt{brew doctor}', deleting any problematic files (using \texttt{sudo rm} to overcome any permission issues). Note that you may wish to do a system backup first.

%=================================================================================================================================
%=== END DOCUMENT ================================================================================================================
%=================================================================================================================================
\end{document}
